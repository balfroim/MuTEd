% Generated from script
    {\footnotesize\begin{longtable}{m{0.55\linewidth}rrr}
        \caption{Questions and Answers by Group}
        \label{tab:questionnaire} \\
        \hline
         & \textbf{Both} & \textbf{PIT} & \textbf{Reneri} \\
        \hline
        \endfirsthead
    
        \hline
         & \textbf{Both} & \textbf{PIT} & \textbf{Reneri} \\
        \hline
        \endhead
    
        \hline
        \multicolumn{4}{r}{\textit{Continued on next page}} \\ \hline
        \endfoot
    
        \hline
        \endlastfoot
    \multicolumn{4}{m{0.9\linewidth}}{\textit{Q1.~What was your knowledge of Java before you started this experiment?}} \\*    No knowledge & 1 (2\%) & 0 (0\%) & 1 (5\%) \\*
    Basic knowledge & 27 (66\%) & 14 (67\%) & 13 (65\%) \\*
    Intermediate knowledge & 11 (27\%) & 5 (24\%) & 6 (30\%) \\*
    Advanced knowledge & 2 (5\%) & 2 (10\%) & 0 (0\%) \\ \hline

    \multicolumn{4}{m{0.9\linewidth}}{\textit{Q2.~What was your knowledge of software testing before you started this experiment?}} \\*    No knowledge & 8 (20\%) & 4 (19\%) & 4 (20\%) \\*
    Basic knowledge & 24 (59\%) & 12 (57\%) & 12 (60\%) \\*
    Intermediate knowledge & 8 (20\%) & 4 (19\%) & 4 (20\%) \\*
    Advanced knowledge & 1 (2\%) & 1 (5\%) & 0 (0\%) \\ \hline

    \multicolumn{4}{m{0.9\linewidth}}{\textit{Q3.~Do you think it is interesting to present the concepts of mutation testing together with the basics of programming?}} \\*    Yes & 17 (41\%) & 9 (43\%) & 8 (40\%) \\*
    Yes but superficially & 20 (49\%) & 10 (48\%) & 10 (50\%) \\*
    No & 4 (10\%) & 2 (10\%) & 2 (10\%) \\ \hline

    \multicolumn{4}{m{0.9\linewidth}}{\textit{Q4.~What could be the consequences of the use of mutation testing by novice programmers?}} \\*    Better programs & 10 (24\%) & 3 (14\%) & 7 (35\%) \\*
    More competent programmers & 6 (15\%) & 2 (10\%) & 4 (20\%) \\*
    Better programs and more competent programmers & 24 (59\%) & 15 (71\%) & 9 (45\%) \\*
    Neither & 1 (2\%) & 1 (5\%) & 0 (0\%) \\ \hline

    \multicolumn{4}{m{0.9\linewidth}}{\textit{Q5.~Do you consider classical testing tools (JUnit) to be useful for teaching programming fundamentals?}} \\*    Yes & 21 (51\%) & 10 (48\%) & 11 (55\%) \\*
    Yes, but only with basic functionality & 17 (41\%) & 9 (43\%) & 8 (40\%) \\*
    No & 3 (7\%) & 2 (10\%) & 1 (5\%) \\ \hline

    \multicolumn{4}{m{0.9\linewidth}}{\textit{Q6.~Do you consider mutation testing tools to be useful for teaching the fundamentals of programming?}} \\*    Yes & 15 (37\%) & 7 (33\%) & 8 (40\%) \\*
    Yes, but only with basic functionality & 21 (51\%) & 12 (57\%) & 9 (45\%) \\*
    No & 5 (12\%) & 2 (10\%) & 3 (15\%) \\ \hline

    \multicolumn{4}{m{0.9\linewidth}}{\textit{Q7.~Considering your background so far (without taking this presentation into account), you feel that the concepts of software testing have been:}} \\*    Fairly well presented & 14 (34\%) & 8 (38\%) & 6 (30\%) \\*
    Insufficiently presented & 26 (63\%) & 13 (62\%) & 13 (65\%) \\*
    Not presented & 1 (2\%) & 0 (0\%) & 1 (5\%) \\ \hline

    \multicolumn{4}{m{0.9\linewidth}}{\textit{Q8.~Do you think that using software testing tools for learning purposes could be useful for creating good programming habits?}} \\*    Yes & 41 (100\%) & 21 (100\%) & 20 (100\%) \\*
    No & 0 (0\%) & 0 (0\%) & 0 (0\%) \\ \hline

    \multicolumn{4}{m{0.9\linewidth}}{\textit{Q9.~Do you think that creating test cases through mutation testing is useful for improving the learning ability of novice programmers?}} \\*    Yes & 39 (95\%) & 21 (100\%) & 18 (90\%) \\*
    No & 2 (5\%) & 0 (0\%) & 2 (10\%) \\ \hline

    \multicolumn{4}{m{0.9\linewidth}}{\textit{Q10.~How did you find creating tests manually, without the help of a tool?}} \\*    Easy in general & 13 (32\%) & 8 (38\%) & 5 (25\%) \\*
    Difficult, especially with regard to the completeness of my tests (sufficient code coverage) & 20 (49\%) & 9 (43\%) & 11 (55\%) \\*
    Difficult, especially to follow a logical order in the design of test cases & 8 (20\%) & 4 (19\%) & 4 (20\%) \\ \hline

    \multicolumn{4}{m{0.9\linewidth}}{\textit{Q11.~What is your perception of software testing after applying mutation testing to your tests?}} \\*    It has changed the way I design tests & 6 (15\%) & 2 (10\%) & 4 (20\%) \\*
    This allowed me to discover parts of the code that were not sufficiently tested & 30 (73\%) & 15 (71\%) & 15 (75\%) \\*
    The mutants do not seem to me to be particularly useful for improving the quality of my tests & 5 (12\%) & 4 (19\%) & 1 (5\%) \\ \hline

    \multicolumn{4}{m{0.9\linewidth}}{\textit{Q12.~The reports generated by the tool used in the second session:}} \\*    Were sufficiently understandable & 39 (95\%) & 20 (95\%) & 19 (95\%) \\*
    Lacked comprehensibility but were still usable & 2 (5\%) & 1 (5\%) & 1 (5\%) \\*
    Were not understandable enough to be usable & 0 (0\%) & 0 (0\%) & 0 (0\%) \\ \hline

    \multicolumn{4}{m{0.9\linewidth}}{\textit{Q13.~Compared to your original self-assessment, you feel:}} \\*    You have assessed yourself correctly & 26 (63\%) & 13 (62\%) & 13 (65\%) \\*
    You have overestimated yourself & 11 (27\%) & 6 (29\%) & 5 (25\%) \\*
    You have undervalued yourself & 4 (10\%) & 2 (10\%) & 2 (10\%) \\ \hline

    \multicolumn{4}{m{0.9\linewidth}}{\textit{Q14.~From a practical point of view, mutation testing:}} \\*    Is very useful & 32 (78\%) & 17 (81\%) & 15 (75\%) \\*
    Is very useful but not comfortable to use & 8 (20\%) & 4 (19\%) & 4 (20\%) \\*
    Does not compensate for the effort required to use it & 1 (2\%) & 0 (0\%) & 1 (5\%) \\ \hline

\end{longtable}}
